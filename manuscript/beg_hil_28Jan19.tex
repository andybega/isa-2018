\documentclass[11pt]{article}
\usepackage[top=1in, bottom=1in, left=1in, right=1in]{geometry}
\usepackage{setspace}
\usepackage{amsmath}
\usepackage{amssymb}
\usepackage{url}
\usepackage{graphicx}
\usepackage{natbib}
\usepackage[    pdftex,                     hyperfootnotes=true,       colorlinks=true,            citecolor=black,             linkcolor=black,             urlcolor=blue,              breaklinks=true             ]{hyperref}

\parskip=0pt
\parindent=30pt
\begin{document}

\date{\today}
\title{Examining Repressive and Oppressive State Violence using the Ill-Treatement and Torture Data}
\author{Andreas Beger\thanks{Predictive Heuristics. email: \href{mailto:adbeger@gmail.com}{adbeger@gmail.com}.} \\
Daniel W. Hill, Jr.\thanks{Assistant Professor, Department of International Affairs, University of Georgia. email: \href{mailto:dwhill@uga.edu}{dwhill@uga.edu}.}}

\maketitle 

\begin{abstract}

\end{abstract}

\clearpage
\setcounter{page}{1}

\doublespace

\section*{Introduction}
It is a well-established fact that state violence is less common in democracies than in non-democracies. However, it is also well established that there are limits to democracy's pacifying effect. Motivated in part by the US government's use of torture in the years after the 2001 terror attacks, several studies find that democracies are no less likely than non-democracies to torture during periods of violent dissent \citep{Davenport2007AR,Davenport2007,DavenportMooreArmstrong2007,ConradMoore2010}. There are also more current examples that illustrate the limits of democracy in preventing state violence. Though it is not a new phenomenon, in recent years there have been numerous highly publicized cases of police officers in the United States using excessive and even lethal force against victims who, in many cases, clearly pose no immediate threat and are suspected of only minor offenses. In one incident, a person was fatally shot while fleeing on foot after an officer stopped him for driving with a broken brake light. In this case the officer was (eventually) convicted of a criminal offense,\footnote{\url{https://www.washingtonpost.com/news/post-nation/wp/2017/12/07/former-south-carolina-police-officer-who-shot-walter-scott-sentenced-to-20-years/}} but criminal penalties are rare in such cases. Of course, this kind of state violence occurs in other democratic countries. Similar events in the Philippines have recently attracted international media attention. As part of a push to eliminate illegal drugs, President Rodrigo Duterte has overseen a campaign of extralegal killings that has claimed thousands of victims over the past few years. Though three members of the national police were recently convicted for the killing of a 17 year old high school student, as is the case in the US, impunity is the more common outcome. Unlike the US government's use of torture as a means of interrogation after the 2001 attacks, state violence in these more recent cases is not motivated by concerns about national security. The victims in these incidents are not targeted because they are viewed as a threat to the government's authority. In some cases they are suspected of criminal activity. In other cases the motive is unclear. This kind of violence, which occurs frequently in some democracies, further demonstrates that democratic institutions are not a panacea to the problem of state violence \citep{Moore2010}.  

Recent research suggests that the events recounted above are not anomalies. While there is overwhelming evidence that repressive violence is less common in democracies, the empirical record is less clear where it concerns state violence that is not politically motivated. \citet{Haschke2018} finds that the relationship between democracy and torture depends on the identity of the victim. When considering cases where the victims are not political dissidents, torture is no more or less common in democracies than it is in non-democracies. Similarly, \citet{JacksonHillHall2018} report that several measures of democracy and judicial constraints are unrelated to torture that targets victims other than dissidents. These findings suggest that the relationship between democracy and state violence requires further examination, and that another caveat must be added to the claim that democratic institutions can constrain the state's violent tendencies.     
   
We follow recent work that examines violence that has no clear repressive purpose. We use the term ``repressive'' to describe state violence that targets organized political opposition or that is meant to raise the cost of organized resistance \citep{Bisselletal1978,Tilly1978, Goldstein1978, StohlLopez1984,Davenport2007AR}. We refer to state violence unrelated to political challenges as ``oppressive'' violence \citep{Bisselletal1978}. Drawing this distinction is crucial, we think, when analyzing the relationship between political institutions and state violence. Most existing research cannot speak to the cases of oppressive violence described above. That is because the majority of explanations for state violence focus on repression, and so view state violence primarily as a tool for minimizing domestic political threats \citep{Haschke2018}.\footnote{See, e.g. \citep{Davenport1995,Poe2004,Pierskalla2010,Ritter2014,RitterConrad2016}.} An exclusive focus on repressive violence leaves us in a poor position to explain why someone would be summarily shot for fleeing after a minor traffic violation, or for purportedly selling drugs. 
%However, many recorded cases are more oppressive in nature, as the victims are not engaging in any activity that could be perceived as a challenge to the state's political authority. 
Explanations for violence that assume it is intended to diminish domestic political threats do not apply to these cases, which means this type of violence likely requires a different explanation. The results in the studies cited above suggest that this is the case.  

In this study we draw on existing work to explain why democracy may have little effect on oppressive violence. Democratic institutions are thought to limit abuse because they make political leaders accountable to the public and make it easier for the public to coordinate opposition to state abuse, broadly defined. However, public backlash is unlikely in response to oppressive violence since the victims are likely to belong to groups with which the average member of the public is unlikely to identify. We also expand the scope of our inquiry beyond political institutions and examine several conditions thought to be related to oppression. Prominent studies identify coercive agents' discretion over detainees as a condition that is related to violent abuse \citep{Rejali2007,ConradMoore2010}. As such, laws intended to prevent arbitrary or lengthy state-imposed detention should be associated with less frequent abuse. We also consider the relationship between oppressive violence and social hierarchies based on ethnicity or class. The presence of a pronounced social hierarchy is also thought to be an important precondition for oppressive violence \citep{Rejali2007} and has been only minimally examined in previous quantitative, cross-national work. In order to examine patterns of repressive and oppressive violence separately we leverage the Ill-Treatment and Torture (ITT) specific allegations data \citep{ConradHaglundMoore2014}, which contains information about thousands of individual allegations of state torture. Whereas previous data on government violence did not distinguish between repressive and oppressive violence, the ITT data include information on the identity type of the victim, which allows us to examine the abuse of dissidents separately from other types of victims. 

In the next section we briefly review the quantitative literature on state violence and summarize what it suggests about democracy's role in constraining the state. In short, explanations for state violence and relevant findings indicate that democracy is more strongly related to repressive than oppressive violence. We then discuss existing explanations for oppressive violence and identify several conditions that should be associated with its use. Following this discussion, we examine the ITT data to establish that violence against victims other than dissidents is relatively common and does not necessarily accompany violence against dissidents. We then conduct a statistical analysis to determine whether the factors we identify are related to, and can help predict, both kinds of state violence.  

\section*{Political Institutions and State Violence}

Since the 1980s, political scientists have been collecting and analyzing data on state violence, and there is now a vast literature on the topic. One of the strongest findings in this literature is that democratic political institutions are negatively associated with violence, with the caveat that democracy has a smaller constraining effect in the presence of violent dissent \citep{Davenport2007AR,Davenport2007,DavenportMooreArmstrong2007,ConradMoore2010}. ``Democracy'' most often refers to a combination of mass participation in politics and meaningful inter-group competition \citep{Dahl1971}. The most straightforward argument for why democracy would reduce abuse is that elected officials in democracies are agents of the public who can be removed from office. Competitive elections provide a low-cost way for the public to get rid of leaders who instigate or allow widespread abuse \citep[E.g.,][]{PoeTate1994}.

%Apart from accountability to the public, another reason political competition may reduce violence is that regular leadership turnover reduces elites' incentive to seize or maintain power through violence. This is because regular turnover lowers the costs of losing office; today's losers may be tomorrow's winners. The outcome of a violent political contest, on the other hand, is often highly uncertain. When it is sufficiently uncertain, the expected cost of using violence to retain power exceeds the cost of peacefully conceding \citep{Przeworski1991}. 

A related line of argument calls attention to how certain political institutions aid the public in coordinating a response to government abuse. Government abuse is broadly defined to include actions such as arbitrary arrest or confiscation of property, torture, violations of constitutional law, and violations of existing rules about political succession. Citizens who wish to oppose such actions face a coordination problem. Opposition must be widespread to be effective, but due to uncertainty about how others will respond and the potentially high cost of participating in political action, mobilizing widespread opposition is difficult. In this strand of the literature elections are also viewed as useful for this purpose \citep[E.g.,][]{Fearon2011}, but most of the focus is on formal protection for individual rights and strong/independent courts of law \citep{NorthWeingast1989,Weingast1997,Vanberg2005,ElkinsGinsburgMelton2009,PowellStaton2009,Melton2013}. Formal rules (e.g., constitutions) aid coordination by providing a common definition of ``abuse,'' which can reduce uncertainty about how others will respond to particular government actions. Powerful constitutional courts make coordination less difficult still, because they provide information about abuse to the public. Such courts provide a prominent, visible signal that abuse has/has not occurred. All of this suggests that formal protections for individual rights, and strong courts of law, should be associated with less frequent government abuse. 

However, there are reasons to doubt that these mechanisms operate in the case of oppressive violence. For one, the public will not necessarily respond to revelations of violent abuse by withdrawing their support for the current government or ruling party in the next election. Building on Walzer's (1973) \nocite{Walzer1973} argument about ``dirty hands,'' \citet{Moore2010} argues that participation/competition is a relatively weak constraint on violence that targets minority groups. In democracies, the public prefers political leaders to be hypocrites. They hold leaders  accountable for providing public security, and expect them to commit human rights abuses in the name of security, but also to publicly deny that any abuse has occurred. Moore goes further and argues that, since torture typically targets small, marginalized groups who the general public perceives as threatening \citep{Rejali2007}, elected leaders pay no political cost even if its use becomes known to the public. 

%Przeworski's (1991) argument quite clearly applies only to violence that is used to seize or maintain political power. Settling political contests with violence is risky, but there is nothing in this account that suggests that competition creates incentives for political elites to oppose the abuse of groups beyond political opponents. 
Arguments about electoral accountability and institutions as focal points for coordination are also more suited to explaining repressive violence. These arguments are closely connected; in both cases, the mechanism that curbs physical abuse is the anticipation of public backlash in response to abuse. Elections, constitutions, and courts of law provide information that allows the public to determine whether abuse has occurred, which creates an opportunity for mass action to remove abusive governments. In this framework, the public has an interest in removing leaders who violate the social pact because abuses of authority are detrimental to their welfare. Moore's (2010) point that the general public does not particularly care about the welfare of the groups most likely to be victims of torture dovetails with one of the insights from \citet{Weingast1997}. The public is not an undifferentiated mass, but consists of multiple groups with different interests, some of whom may not be affected by the abuse of others or may actually benefit from it. If abuse is targeted rather than indiscriminate, then for an individual member of the public to respond it must be the case that they anticipate being a likely victim of abuse in the future. In this eventuality they may require the help of today's victims, which gives them incentive to oppose the abuse of others in the present. This argument provides another reason to believe that these institutions are of limited usefulness for preventing oppressive violence. Consider first repressive violence, which in many cases targets people for exercising basic political rights recognized under international law, i.e.\ openly supporting political opposition groups. It seems plausible that the mechanism described above can effectively prevent this kind of abuse since ``supporter of an organized political group'' is a category to which many members of the public belong, if ``support'' includes merely openly agreeing with the policy preferences of a particular group. As a consequence, it is easy for an individual to imagine that violence against supporters of political groups may affect them. Many people can imagine themselves being in the political minority, voting for a losing candidate, etc. After all, given regular turnover in leadership even supporters of the current government must keep in mind that there is some chance that they will be in the ``opposition'' in the future. 
%This is precisely the point Przeworski makes, applied to the public rather than political elites. 
%These arguments are partly about the maintenance of democracy itself; all of these institutions reduce a government's incentive to forego peaceful competition and forcibly subdue their political opponents.  
On the other hand, these institutions only go so far towards reducing abuse that targets groups who members of the public do not readily identify with. Oppressive violence targets victims for reasons other than their political preferences, and victims of such violence typically belong to groups that are, by definition, a minority of the population: criminal suspects, immigrants, members of ethnic minorities, etc. Institutions that provide a low cost mechanism for opposing abuse will not matter when the public has no incentive to respond to abuse. And, since the average member of the public is more likely to identify as supporter of a political group than they are as, e.g., an immigrant, violence that targets the latter group is less likely to trigger a response from the average member of the public. Knowing this, political leaders will not be especially concerned about backlash in the wake of abuse that targets these groups. 

This claim is consistent with findings from several recent studies. \citet{ConradHillMoore2018} argue that competitive elections do not create political costs to leaders if the public becomes aware of torture. As a result, leaders have no reason to encourage state agents to use ``stealth'' torture techniques that leave no scars on victims' bodies and are less easily detected. They present evidence that competitive elections are associated with {\em more} allegations of scarring torture. Though this study does not speak to the distinct nature of repressive and oppressive violence, it casts doubt on the mechanism through which democracy is thought to limit state violence. Haschke's (2018) study calls attention to the fact that non-politically motivated state violence is relatively common in both democracies and non-democracies, which does not comport with the literature's nearly exclusive focus on repressive violence. Using the ITT data, he finds that democracy is related to politically motivated violence (i.e., violence against dissidents), but is unrelated to some kinds of torture, in particular torture that targets individuals from ``marginalized'' social groups for reasons unrelated to dissent. \citet{JacksonHillHall2018} use ITT to analyze the relationship between torture and two  institutions associated with democracy: competitive election and judicial constraints. They examine torture by the police against dissidents, criminal suspects, and marginalized individuals, and find that while elections and judicial constraints are strongly related to the torture of dissidents, they are only weakly related to the torture of criminals, and are not meaningfully related to the torture of marginalized individuals. Thus the available evidence suggests that democracy is a less effective restraint where oppressive violence is concerned.       

Concerning repressive violence, while democracy appears to discourage its use, this is partly because the use of violence against political opposition groups constitutes a violation of democratic principles, so that regimes who use repressive violence will be classified as less democratic than regimes who do not \citep{Hill2016}. This is another reason it is necessary to distinguish between repressive and oppressive violence: when examining the effect of democracy on state violence, we must be careful to exclude violence that may disqualify a state from being considered a democracy. Arguably, most existing analyses use indicators of democracy and state violence that overlap to some degree. 

This discussion suggests that arguments about democracy's ability to constraint state violence may not provide much insight into patterns of oppressive violence. In the next section we discuss explanations for oppressive violence and identify several broad conditions that should be related to its occurrence. 

%\section*{extra stuff} 

%The discussion above suggests that certain political institutions will be more effective at curbing repressive violence than oppressive violence. To reiterate, this is because the mechanism linking these institutions to reductions in violence depends on the willingness of members of the public to take action against abuse that does not directly affect them. This is much more likely to be the case when victims of abuse belong to a social category with which a large swath of the public identifies. In turn, this is more likely when violence is repressive rather than oppressive.    

%Use specific allegations data, which we know less about than the LoT data. Some results contradict previous findings.  

\section*{Explanations for Oppressive Violence}

While most political science research on state violence is concerned primarily with repression, there are notable exceptions. \citet{Rejali2007} develops three explanations for how torture can become common in democratic countries. Only one of these, his ``national security model,'' resembles the standard account in which states use violence to combat political threats. Discussing historical examples, Rejali notes that democracy has often failed to prevent torture during periods of violent conflict. In this context responsibility for security may be delegated to the military, who will often use torture for the purposes of interrogation. Systematic examinations of state violence in the context of violent dissent suggest this is a general pattern \citep{Davenport2007,Davenport2007AR,DavenportMooreArmstrong2007,ConradMoore2010}. Rejali's other explanations for torture in democracy are not intended to apply to repressive violence. One is that police often rely on confessions by criminal suspects to secure convictions in court. Police use various methods to pressure suspects to confess, and sometimes resort to physical violence. Rejali points to Japan as an example of a democracy with a legal system whose features create many opportunities and incentives for police to use torture. In particular, courts rely heavily on confessions as evidence, and police have wide discretion over where and how long they detain suspects. With the permission of a court, police may hold suspects in detention for up to 23 days without filing criminal charges. Suspects are often detained in specialized facilities designed to hold people still under investigation, and are subjected to lengthy interrogation. Consequently, torture during pre-trial detention is relatively common.\footnote{See the Committee Against Torture's \href{https://www.refworld.org/publisher,CAT,,JPN,,,0.html}{Concluding Observations on Japan's second periodic report}.} In addition to torture to obtain confessions in criminal investigations, Rejali discusses what he calls torture for ``civic discipline.'' This kind of torture occurs in (usually) urban environments where police provide ``law and order'' by using violence to demarcate social-geographic boundaries. This is often for the purpose of keeping  ``undesirables'' out of certain areas or neighborhoods, or to simply remind them that they are of low status and therefore vulnerable. Violence for civic discipline targets people who have the formal, or informal, status of quasi- or non-citizens. Protected citizens ignore police violence as long as they believe it contributes to public (i.e., their) safety. 

\citet{Haschke2018} provides a lengthy discussion of the distinction between repressive and oppressive (non-political) state violence, and points out that many recorded instances of violence are not repressive in nature. He conceptualizes state violence through a principal-agent framework, an approach adopted previously by \citet{ConradMoore2010}. This approach treats coercive bureaucracies as agents of the executive. Following the general insights of principal-agent models, these authors argue that if security forces and agents responsible for detainees have different preferences than the executive and are not closely monitored, then they are more likely to use violence in pursuit of organizational or self-interested goals. \citet{ConradMoore2010} present an argument about how political institutions affect leaders' incentives to expend resources monitoring coercive bureaucracies. Haschke points out that explanations for repression have in mind violence that helps the executive accomplish her goal of remaining in office, but oppressive violence does not necessarily contribute to this goal. He argues that we must examine the motivations of agents themselves to explain this kind of violence and discusses various motives, including those outlined by Rejali, as well as endemic corruption that creates opportunities to use violence for monetary gain. Like \citet{ConradMoore2010}, he also emphasizes the importance of monitoring. The conditions he identifies as contributing to agency loss include effective civilian control of the military/police, and impunity for perpetrators. 

Based on this discussion we identify two broad conditions that should be associated with oppressive violence. One is how much discretion coercive agencies have over decisions about detention, including the conditions that justify detention and how long someone should be detained. Following arguments based on the principal-agent framework, where there are few checks on the ability of security forces to hold people in their custody, agents will have more opportunities to engage in abuse. Several relatively common legal rules are intended to prohibit prolonged or arbitrary detention, often by giving courts rather than police discretion over decisions about detention. Such rules should help prevent abuse by the police, including the torture of criminal suspects to extract confessions. Laws that prevent police from holding suspects in custody without a criminal charge, laws that allow for the possibility of pre-trial release, the right to a writ of {\em habeas corpus}, and laws granting defendants the right to a fast trial create legal obstacles to lengthy or unjustified detention and will potentially reduce opportunities for abuse. 

The other condition that is especially relevant to oppressive violence is the existence of a clear social hierarchy, which is a precondition for violence for ``civic discipline'' as discussed by \citep{Rejali2007}.\footnote{\citet{Rejali2007} also mentions the privatization of security, the presence of immigrants, and a lack of resources to monitor state agents as determinants of violence for civic discipline.} \citet{Haschke2018} examines social hierarchies based on class, and measures the strength of social hierarchy using income inequality. It is certainly plausible that high income inequality would create demand for the kind of policing described by Rejali. Where income differences are relatively steep wealthier citizens may perceive less wealthy, quasi-citizens as potentially dangerous, and state institutions are likely more responsive to the demands of wealthier citizens. 

In addition to class, social hierarchies based on ethnic identity should be relevant to oppressive violence. Ethnicity is another distinction that often forms the basis of social hierarchy and is used to determine who counts as a protected citizen. The literature on ethnic identity, politics, and conflict is vast, and anything close to a comprehensive discussion of it is beyond the scope of this paper. Some of its insights are relevant to oppressive violence, however, and we briefly sketch the connection here. Scholars of ethnic politics view ethnicity as a set of social categories/identities that can be used for political mobilization, including mobilization for violence. This is most likely when a particular ethnic identity is accompanied by a readily available narrative that emphasizes grievances based on a shared experience of discrimination and exclusion. Historically, ethnicity has often served as the basis for social, economic, and political discrimination and exclusion. In some cases states treat one identifiable ethnic group as ``rightful'' members of the nation-state who have the status of protected citizens, while members of out-groups are perceived, and sometimes given the formal status of, quasi-citizens. In these cases the dominant group is likely to benefit disproportionately from the state's policies. It is well established that political exclusion based on ethnicity contributes to violent dissent and civil war \citep{birnir2006ethnicity,cederman2013inequality}.
%\footnote{ \citet{birnir2006ethnicity} argues that, at least in new democracies, exclusion may be a strategic consideration rather than a reflection of genuine hostility between groups or beliefs in ethnic superiority.} 
Ethnicity-based exclusion and inequality should also be related to oppressive violence, as it clearly indicates a social hierarchy where one or more groups are viewed as less than full citizens and will likely not receive the state's full protection. State agents may face few consequences for abusing members of these groups, so abuse is more likely for reasons explained by the principal-agent framework. But it is important to recognize that oppressive violence under these conditions is not necessarily the result of a principal agent problem. Abuse of a minority group could occur systematically as a direct result of policy. As discussed by \citet{Rejali2007}, leaders can build public support through ``law and order'' policies that often entail more policing and abuse of marginalized  communities, including low-income and ethnic minority communities. In that sense, state agents may be accomplishing executive goals through the use of oppressive violence. Additionally, social hierarchies, especially those based on ethnicity, are likely to be characterized by dominant groups with beliefs in their superiority that provide a normative justification for violence in response to slight infractions \citep{sidanius2001social}. All of this leads us to argue that where social hierarchies are especially pronounced oppressive violence is more likely. 

In the next section we discuss the ITT data and how it can be used to distinguish between repressive and oppressive violence. We also present some descriptive statistics and patterns that speak to the relative prevalence of these kinds of violence, and their dissimilarity. 

%Birnir argues that exclusion in new democracies is a strategic choice, but clearly it is sometimes accomplished through violence and also maintained through violence. Violence need not be reactive. Rejali's civic discipline model (which is about democracies) suggests violence against marginalized groups occurs in absence of dissent or even political organization. 

\section*{Repressive and Oppressive Violence in the ITT data} 
%He argues that its high frequency is the result of an excessive reliance on confessions rather than physical evidence in criminal trials, and legal rules that allow police to easily prolong pre-arrest detention periods. A combination of wide discretion about detainee treatment and professional incentives to produce evidence results in routinized torture. 
%His ``juridical model also seems to apply well to cases like Jon Burge's use of torture over two decades in Chicago. Burge and others, convinced of suspects' guilt and eager to produce evidence, used torture to extract confessions that, in many cases, led to convictions in court. Some of these convictions were overturned (years later) after investigation revealed the role of torture in the confessions. In Chicago police designed an extralegal system to circumvent rules prohibiting prolonged detention. 
The indicators of state violence used in most studies partly measure violence that is more oppressive than repressive in nature. The two most commonly used measures, the CIRI physical integrity rights index (Cite CIRI) and the Political Terror Scale (Cite PTS), are created through a content analysis of annual human rights reports from the US State Department, Amnesty International, and in the case of PTS, Human Rights Watch. Though these reports give call attention to politically motivated abuses, they also include information about violent abuse that has no clear connection to dissent \citep[][pp.\ 13--16, 90--91]{Haschke2018}. 
%For example, consider Amnesty International. Though the organization was created with the goal of creating awareness of political prisoners, it has been conducting a ``Campaign Against Torture'' since 1972. The organization's broad goal is to publicize human rights abuses where it finds credible evidence of abuse having occurred. As such, AI's reports are not limited to politically motivated violence, but contain allegations of abuse that targets a diverse range of people. 
For example, the 1995 Amnesty report for Brazil discusses torture in police stations and prisons. The legislature's Human Rights Commission campaigned that year to make torture a criminal offense, and legislative hearings at the state level that year ``confirm[ed] allegations that torture continued to be widespread and a common method of extracting information from criminal suspects.'' The report goes on to recount two cases where victims of torture died from their injuries. One was a domestic servant suspected of stealing money from her employer. Another was suspected of drug-related offenses. As another example, that same year in Bulgaria, Amnesty reported numerous allegations of torture and ill-treatment by the police, noting that ``many of the victims were Roma.'' In one case a victim was found dead in the street, wearing handcuffs. A witness reported seeing a police officer the previous day, apparently drunk, beating the victim with a piece of wood. In this case no motive for the killing was reported. It is not at all difficult to find similar cases in other reports. 

Because commonly used indicators, which are all based on these reports, provide a single score for each state, a poor score reflects a mixture of repressive and oppressive violence that cannot be separated. In order to separate out these distinct kinds of government violence, we take advantage of a relatively new data collection effort: the Ill-Treatment and Torture Data Collection Project \citep{ConradHaglundMoore2014}. The ITT data catalogues allegations of torture by Amnesty International from 1995-2005 (inclusive). These allegations come from AI's annual reports, action reports, and press releases. This study relies on ITT's Specific Allegations data, which contains information about 16,431 individual allegations of state torture. In each event there is a victim who is in the custody of a state agent (the perpetrator), and an act of violence that meets the international legal definition of torture.\footnote{The definition comes from The United Nations' Convention Against Torture and Other Cruel, Inhuman or Degrading Treatment or Punishment (CAT).} Whereas previous data on government violence did not distinguish between repressive and oppressive violence, the ITT data include information on the identity type of the victim, which allows us to examine the abuse of dissidents separately from other types of victims. 

The ITT uses a typology of victims based on the information contained in AI's reports. Though Amnesty International began with the goal of calling attention to political prisoners, it has been conducting a ``Campaign Against Torture'' since 1972. The organization's broad goal is to publicize abuse where it finds credible evidence of abuse having occurred. As such, AI's reports are not limited to politically motivated violence, but contain allegations of abuse that targets a diverse range of people. Dissident is one of the victim types used in the ITT data, but there are several others, including criminal, member of a marginalized social group, state agent, prisoner of war, and unstated, meaning the allegation does not identify a victim type.\footnote{In about one quarter of the allegations in ITT the victim's identity type is not mentioned.} For allegations that list a victim identity, the identity may correspond to more than one category, which is the case in 2,704 allegations (16.5\%). If we split events with multiple victim types into separate observations, this results in 19,422 victim type-allegations. Figure \ref{fig:victim-types} shows the resulting sum total of allegations for each of the six victim types. 

\begin{figure}
\begin{center}
\caption{Total number of allegations in the ITT data by victim type}
\label{fig:victim-types}
\includegraphics[width=.75\textwidth]{../output/figures/allegations-by-victim.png}
\end{center}
\end{figure}

The ITT specific allegations data suggest that oppressive violence is more common than repressive violence. Figure \ref{fig:victim-types} makes it clear that in most of the allegations in the ITT data the victim is not a dissident. If we include only allegations with a unique, identified victim type, dissidents account for 31\% of allegations, and are the second most common victim type after members of marginalized groups (38.8\%).\footnote{The ITT defines a member of a marginalized social as someone who ``is tortured by the state for the purpose of social control (i.e., humiliation or other punishment to establish that [1] her/his behavior was inappropriate and [2] that the state can abuse her/him with impunity), rather than for the collection of information.'' It lists as candidates immigrants, asylum seekers, the homeless, geeks, punks, skinheads, as well as minorities, e.g.\ sexual and national (ITT Codebook, p.\ 25--26).} Criminals account for 27.7\% of these allegations, while POWs and state agents combined account for less than 3\%. If we add to our count of dissident allegations cases where the victim falls into the dissident category and at least one other category, then allegations involving the torture of dissidents make up about 38\% of all of the events in ITT with an identified victim type. Of all of the instances of torture publicized by AI between 1995-2005 where the identify of the victim was mentioned, the victim was not a dissident about 62\% of the time.\footnote{See \citet[][pp.\ 3-4]{Haschke2018}, who reports that allegations where the victims are dissidents make up a minority of all allegations even in autocratic countries.} 

The ITT data do not constitute a census of cases of torture. Its creators encourage users to treat the data as what they are, {\em allegations} of torture, rather than a count of cases of torture.\footnote{See \citet{HillMooreMukherjee2013} for an analysis of the CIRI torture scale that accounts for the process by which AI produces information about torture. See also \citet{ConradHillMoore2018} who analyze ITT using a model that accounts for the fact that allegations represent a subset of all cases of torture.} It should be noted that, as a measure of actual torture, the the ITT data are at least as valid as existing torture scales, which are created partly, and in some cases entirely, from the same set of documents, but record less information about the abuse described in those documents. With this in mind, an examination of the allegations themselves can be instructive for our immediate purpose, which is simply to show that state violence extends well beyond the abuse of dissidents. Taking ITT as an indication, it is plausible that oppressive violence is at least as common as repressive violence. This means that explanations for state violence that focus on the repression of dissent leave many, perhaps most, cases of violence unexplained. 

Despite the fact that perpetrators of oppressive and repressive violence have distinct motives, it is certainly possible that these kinds of state violence tend to occur in similar social, political, and economic contexts and so will be predicted by many of the same factors. If this were the case it may not be consequential to use measures that aggregate the two, as is common practice. Of course, we cannot know whether this is the case unless we disaggregate the two and consider them separately. An examination of the ITT data suggests that the torture of different victims follow quite different patterns. Figure \ref{fig:correlation-matrix} shows the correlations between allegation counts for the four most common victim types. Each histogram summarizes the within-country correlations between a pair of torture victim types on the relevant axes. The text labeled $\bar{r}$ shows the across-country average correlation coefficient for that pair of victim types. These correlations, if we exclude the self-correlations on the diagonal, are fairly close to the global average of .34. So it is not the case that countries that countries that torture dissidents are especially prone to torture marginalized groups and criminals as well. This disconnect is even more pronounced at the country level, as shown by the gray histogram which show the distribution of within-country correlation coefficients for that pair of victim types. About 20\% of the within-country correlations are 0 or negative, i.e.\ increases in allegations of torture related to one kind of victim are associated with {\em decreases} in allegations of torturing another kind of victim. The general point is that allegations of torturing different kinds of victims are only loosely related. In other words, those countries that (allegedly) torture dissidents  do not necessarily (allegedly) torture criminals or marginalized individuals. 

\begin{figure}
\begin{center}
\caption{Summary of within-country correlations between torture allegations of specific victim types. The text in each plot shows the average within-country correlation when we average accross countries. Torture allegations of different victim types are only loosely correlation and in about 20\% of countries negative, indicating lower allegations for one victim type when allegations for another are high.}
\label{fig:correlation-matrix}
\includegraphics[width=.99\textwidth]{../output/figures/allegations-by-victim-pairwise-correlations.png}
\end{center}
\end{figure}

%REWRITE 
In the next section we conduct an analysis to examine whether, as our argument suggests, the conditions that predict each kind of violence are not identical. The divergence discussed above provides reason to believe that the two types of state violence have different determinants. At the very least, it will be useful to know whether the same set of conditions that are known to predict repression reasonably well can also predict non-politically motivated state violence. If they do, then we can safely analyze them together and perhaps develop more general explanations for state violence. If they do not, then more data collection efforts that distinguish between different kinds of violence, and theoretical models that treat them separately, will be necessary.  

\section*{Analysis}

The ITT data cover the years 1995 to 2005, 11 years in total. Our data consist of country-years for independent states during that period, following the Gleditsch and Ward list \citep{gleditsch:ward:1999}. This makes for a total of around 1,600 observations. In the raw allegation data, a single allegation may specify any number of six different victim types. We split such multi-victim allegations into separate allegations for each victim type before counting allegations by country-year-victim type. As discussed above, there are few allegations of torture of POWs and state agents, so we leave these out. We also drop the unstated category from the models below. We analyze allegation counts for dissidents, criminals, and marginalized groups, which are the three most common victim types. We analyze these allegations separately. 

Our analysis uses as covariates measures of electoral competition, institutional constraints on the executive, legal barriers to state-imposed detention, and social hierarchy. Electoral accountability and constraints on the executive are the political institutions that receive the most focus in the literature on state violence \citep[E.g.,][]{Davenport2007}. In contrast to most studies, previous analyses that use victim-specific indicators in ITT fail to find a clear relationship between these institutions and violence against criminals and marginalized individuals \citep{Haschke2018,JacksonHillHall2018}. These studies both use the ITT ``level of torture'' scale from the country-year version of ITT \citep{conrad2013disaggregating} rather than the specific allegations data. We use a binary measure of electoral competition from \citet{cheibub2010democracy} used in \citet{Haschke2018,JacksonHillHall2018}. This makes our results comparable to previous studies. This indicator also uses an operational definition of competition that is less likely to overlap with operationalizations of repressive violence than those used by other measures of competition \citep[See][]{Hill2016}. We take two measures of constraints on the executive from the Varieties of Democracy data (V-Dem) \citep{Vdem}. One captures judicial constraints, and is created from component scales that measures judicial independence and executive compliance with the judiciary. The other indicates the extent to which the legislature exercises effective oversight of the executive. 

For legal barriers to detention we use several indicators from the Comparative Constitutions Project \citep{CCP2014}. For this purpose we use measures of several relevant constitutional provisions. These are all binary indicators coded one if the provisions appears in the national constitution. One of these indicates the inclusion of an explicit reference to due process rights. Another is coded one if the constitution contains a provision for a writ of {\em habeas corpus} to protect individuals from unjustified detention. This encompasses provisions that prohibit arbitrary detention, create a requirement of formal accusation, or arrest based on a warrant or court order. We also include a measure of whether the constitution includes a provision allowing for the possibility of pretrial release. Such provisions include those that ban the imposition of excessive bail, or state that bail cannot be refused or denied without just cause. Another indicator measures whether there is mention of the right to a fast trial, or a trial within a reasonable time. Finally, we include a measure of explicit constitutional bans on torture. 

To measure social hierarchies we rely on two data sources. One is the Ethnic Power Relations data (EPR), from which we take a measure of the proportion of a country's population that belongs to an identifiable ethnic group that is excluded from national politics \citep{vogt2015integrating}. ``Excluded'' means the group is either not represented in the executive branch, is actively discriminated against in public politics, or live in a region that is (possibly {\em de facto}) autonomous from the central government.  The other source we use is the V-Dem data. From this source we use four measures. Two indicate the extent to which political power and influence is distributed evenly across social groups. The first of these considers whether groups defined by class (disparities in wealth) have differential influence. The second considers whether groups defined by ethnicity, language, religion, race, region, or caste. The other two indicators measure whether different class or ethnic groups enjoy the same civil liberties protections, where civil liberties are defined to include access to justice, property rights, freedom of movement, and freedom from forced labor. 

Our regression models are all Poisson models with random intercepts for countries. We include random intercepts 
%for two reasons. The first is that we will evaluate our models in part by their ability to predict outcomes, i.e. model fit, and including random country intercepts improves model fit a lot. 
because, by soaking up between country variation in outcomes with the intercepts, we can be more certain that any associations we find for variables of interest are not spurious correlations driven by other differences between countries. The downside is that, to the extent that the factors we examine (some of which do not vary much across time) really do account for cross-country differences in torture allegations, we are stacking the odds against finding so. 
%The number of observations per group are not very large, 11 years per country at most, which could be problematic both for estimating model parameters and overfitting. The former seems to be more of an issue than the latter, as out of sample accuracy is fairly close to in-sample accuracy. And to the extent that things, some of which don't vary much across time, really do account for cross-country differences in torture allegation levels, we are stacking the odds against finding so. 
We begin by estimating a baseline model that includes (the natural logs of) GDP per capita and population size, both taken from the World Development Indicators. In this model we also include a measure of INGO restricted access from the ITT data, per the codebook \citep[][p.\ 17]{ITTsaguide}, and a measure of internal conflict resulting in $\geq 25$ annualy battle deaths from the UCDP armed conflict data \citep{Themner2014}. We then evaluate the relationships between our variables of interest and the three allegation counts in separate models, where each model includes the covariate of interest plus the controls. 

\begin{figure}
\begin{center}
\caption{Coefficient estimates for Poisson regression models of torture allegation counts by victim type. All models include random country intercepts.}
\label{fig:coefs}
\includegraphics[width=.99\textwidth]{../output/figures/model-coefs.png}
\end{center}
\end{figure}

Figure \ref{fig:coefs} displays the coefficient estimates from all models with 95\% confidence intervals, as well as the global intercepts and the standard deviations of the random intercepts. For control variables, which are included in all models, multiple estimates are displayed. We note first that the coefficients for the binary democracy are all negative and statistically significant. The estimate for the dissident model is largest, followed by marginalized individuals and then criminal suspects. This ordering is consistent with our expectation that electoral competition is more strongly associated with repressive than oppressive violence. However, this result contradicts previous findings, which suggest that democracy has no discernible relationship with the torture of marginalized groups, and only a weak relationship with the torture of criminal suspects \citep{Haschke2018,JacksonHillHall2018}.  

The coefficients from the models that include constitutional provisions to curb arbitrary or lengthy detention are all negative, and significant in all but two cases, indicating that, in general, these provisions are negatively related to allegations of torture. The only exceptions are the coefficients for due process and {\em habeas corpus} in the models using the criminal suspects victim type. The estimates are similar across victim types. This is perhaps not surprising since rules about discretion over detention should apply to anyone in the state's custody, though we would expect some of these to be more relevant to criminal suspects than other kinds of victims.      

The estimates for excluded population from the EPR data are all positive, which is expected. Especially surprising, however, is the fact that the estimate is largest for criminal suspects, followed by dissidents, then marginalized individuals, for which it is not significant. We expected this measure to be associated with allegations concerning the latter victim type. 

The estimates for the V-Dem indicators of social hierarchy are all negative as expected, though two estimates are insignificant: the variables measuring whether political power is shared equally across class and ethnic groups are insignificant in the models for criminal suspects. Other than this the estimates are similar across models, though they are slightly larger (in absolute value) for the marginalized individual models.   

Turning to the measures of executive constraints from V-Dem, estimates for judicial constraints are all negative. The estimate for criminal suspects is not significant, however, which is consistent with the reported result in \citet{JacksonHillHall2018}. The estimate for marginalized individuals is significant, however, which contradicts the finding reported in that study. This in combination with the result for electoral competition suggests that there may be some differences between the ITT country-year data and the specific allegations data that may be worth exploring. For legislative constraints the coefficients are negative and significant for dissidents and marginalized individuals, while the estimate for criminal suspects is slightly positive and insignificant. 

The estimates for the control variables are generally in line with previous findings. Notably, GDP per capita is more strongly associated with allegations of torturing dissidents than with allegations related to the other victim types. The estimates for armed conflict are largest in the case of dissidents and marginalized people, and are insignificant for criminal suspects. 

In addition to count regression models, we evaluate the predictive power of each covariate for each allegation count using a machine learning predictive model called XGBoost. The model is an ensemble of decision trees, where each decision tree is specifically trained to reduce leftover prediction error given the current ensemble prediction. Each component decision tree itself is a very simple model that predicts allegation counts by splitting the input data into several groups based on the values observations have on selected independent variables. The model determines the importance of a particular input variable in predicting the outcome by calculating the reduction in prediction error that results from including that variable in the model. 

Variable importance scores for each variable are displayed in Figure \ref{var-imp}. For each variable the figure shows an importance score for each victim type. Across victim types, the set of variables that add the most predictive accuracy is very similar, though the orderings are slightly different. GDP per capita, population, legislative constraints, judicial constraints, and the ethnic exclusion variables all contribute to relatively large improvements in prediction. The group of V-Dem variables measuring social hierarchy are relatively less important according to the model, and the constitutional provisions variables add little predictive power. For the criminal suspect victim type, GDP per capita and population outperform every variable. Legislative constraints is third, followed by judicial constraints and excluded population (percentage). For allegations involving  marginalized victims, GDP per capita and population are also first and second, followed by the number of exclude ethnic groups, then legislative and judicial constraints. For dissidents, GDP per capita is first, followed by legislative constraints, population, judicial constraints, and the number of excluded groups. Notably, dissident is the only victim type where any variable outperforms GDP per capita and population. Also notable is that only for the marginalized victim category does ethnic exclusion outperform executive constraints. 

Though this set of variables does relatively well at predicting allegations for each victim type, the importance scores for individual variables varies quite a bit across victim types. Legislative constraints especially, but also Judicial constraints, is better at predicting the outcome for dissident victims than for the other two victim types. The ethnic exclusion variables add more accuracy for marginalized individuals than for dissidents or criminal suspects. This latter result is at odds with our regression model results for this variable, which was not statistically significant in the model for marginalized victims. This is likely due to the fact that our regression models include random intercepts, which account for much of the across-country variation in allegations. Coefficient estimates in those models are based partly on within-country variation. Variables that are good at accounting for across-country variation but not within-country variation over time will likely have small coefficient estimates. In contrast, the XGBoost results are likely picking up on the ability of the variables to predict the outcomes across countries, as most of the variation in allegation counts is across rather than within countries.  

\begin{figure}
\begin{center}
\caption{XGBoost variable importance for predictive accuracy.}
\label{var-imp}
\includegraphics[width=.99\textwidth]{../output/figures/xgboost-variable-importance-v1.png}
\end{center}
\end{figure}

%In addition to the coefficient estimates for the variables that interest us, we also consider the impact of adding a variable on model fit, and out of sample fit specifically. This indicates how well any empirical results are to generalize outside the 11-year coverage of the data. To derive out of sample predictiosn we used 11-fold cross-validation, which roughly corresponds to predicting one year of outcomes with the remaining 10 years of data. 

%The last model is a commonly used machine learning predictive model, XGBoost, which provides another benchmark of the levels of prediction accuracy one should expect to be feasible. Hyperparameters for this kind of model are usually tuned via cross-validation. In order to be able to compare the cross-validation predictions to those from the count models, we did not do this and instead left hyperparameters at their default values.

\section*{Conclusion}

Based on our analysis, we draw three broad conclusions about the differences between repressive and oppressive violence. First, the political institutions that receive the most attention in the literature, electoral competition and executive constraints, appear to be better at protecting dissidents from physical abuse than other kinds of victims. That is, they are more reliably associated with repressive violence than oppressive violence. On the one hand, some political institutions are associated with oppressive violence in our analysis. Competitive elections are related to allegations involving all the victim types we examined, including those that we take as indicators of oppression. Judicial constraints and legislative constraints are both related to allegations of torturing marginalized individuals. Legislative and judicial constraints also predict allegations of oppressive violence well relative to most of the factors expected to influence oppression. On the other hand, these institutions, in particular competition and legislative constraints, are more strongly associated with repressive violence than oppressive violence. They are also better at predicting repressive violence than oppressive violence. This supports the claim that explanations for state violence that focus on political institutions are better suited to explaining repressive violence than oppressive violence. It is telling that the only variable not outperformed in terms of prediction by both GDP per capita and population size is legislative constraints for the dissident victim model. 

Second, the conditions we identify as important influences on oppressive violence have less explanatory power than expected. The most promising result on this count is that ethnic exclusion is associated with violence against criminal suspects in our regression models, and predicts violence against marginalized groups better than political institutions. Constitutional provisions related to detention are associated with oppressive and repressive violence, but do not predict these outcomes well. However, we use these provisions only as proxy measures for oversight and monitoring of state detention. More direct measures of that concept, though difficult to come by, should be useful for future research.     

Finally, we note that the decision to examine different kinds of state violence separately should be guided by the explanation for violence one wishes to test. Because most existing explanations have in mind repressive violence, it is not appropriate to examine oppressive violence to determine if they are incorrect. However, some arguments are more general. The principal-agent framework discussed above, for example, suggests that allocating resources to monitoring coercive agents, and punishing those responsible for abuse, should reduce both kinds of violence.   

%On the other hand, democracy. We've explained why it has to be related to abuse of dissidents. But take case where operational definition doesn't overlap w/ repression. E.g., legislative constraints. When opposition can check executive, abuse of dissidents is less likely. Makes sense that the opposition would oppose abuse of gov't opponents, since they are in a similar position. But not same w/ criminals. In fact, things that matter for other two don't matter for criminals, suggesting this may be most difficult group to protect.   

\clearpage
\begin{singlespace}
\bibliographystyle{apsr}
\bibliography{beg_hil}
\end{singlespace}

\end{document}

\begin{figure}
\begin{center}
\caption{Model fit and predictive accuracy. Mean absolute error (MAE) and root mean squared error (RMSE) are calculated using out of sample predictions obtained via 11-fold cross-validation.}
\label{fig:fit}
\includegraphics[width=.99\textwidth]{../output/model-fit-plot.png}
\end{center}
\end{figure}

\begin{figure}
\begin{center}
\caption{Predicted versus observed values for the country intercepts-only model (model 1), as an example of what the predictions for all models look like.}
\includegraphics[width=.99\textwidth]{../output/mdl1-y-vs-yhat.png}
\end{center}
\end{figure}

The second count model adds GDP per capita and a binary democracy indicator. These are retained as control variables in the subsequent models. The binary democracy indicator is derived from the Cheibub et al democracy data. This codes six types of regimes, three each for democracies and dictatorships. We collapse the subtypes into a single binary democracy/dictatorship indicator. Adding these two variables significantly improves fit across all metrics we consider in Figure \ref{fig:fit}.

\textbf{Judicial independence}. The first measure we examine is latent judicial independence, a continuous variable ranging from 0 to 1. Bivariate plots of judicial independence in Figure \ref{fig:lji-bivariate} show that it is related with a decrease in allegations of dissident torture, but it is also apparent that the effect size is not very large in relation to the overall variation in allegation counts. The blue lines show fitted regression lines, and for dissident allegations the coefficient is negative and statistically significant with $p<0.05$. 

Judicial independence is strongly related to regime type, being higher in democracies than dictatorships in general, and highest in parliamentary democracies specifically. In model three, we estimate the effect of judicial independence when controlling for democracy, and the effect on dissident torture remains negative and significant. Judicial independence is associated with a reduced number of dissident torture allegations, in addition to any protective effects provided in democracies generally. In terms of model fit however, adding judicial independence does not clearly improve predictive accuracy. The MAE and RMSE slightly decrease, but not by enough to justify the added model complexity if we look at the information criteria AIC and BIC. 

Some positive evidence for judicial independence is provided by the XGBoost model results. It is common to evaluate the input variables for a machine learning model by how much they effect the model's predictive accuracy--``variable importance'', and these numbers are shown in Figure \ref{var-imp}. The higher the variable importance, the more a variable matters for predicting the outcome, and latent judicial independence ("LJI") is fairly high up by average importance across the three outcomes, fifth after GDP, country population, \% of GDP from rents, and GDP per capita. It is also notable that LJI predicts criminal torture better than any of the democracy variables we examine. 

Since the XGBoost model does not include random country intercepts and thus reflects a variable's ability to predict between country differences more than the count models with random intercepts do, we can tentatively conclude that the model fit tells us that (1) increases in judicial independence appear to be related to a decrease in torture allegations of dissidents, but (2) also that judicial independence has a relatively stronger ability to explain between country differences in torture allegations. 

\textbf{Democracy}. Democracy itself, if we look at the estimates across all models and outcomes, appears to be robustly related to a decrease in torture allegations against victims of all three kinds. That probably matches prior expectations, but also is a finding that appears simpler than it is. The relationship between democracy and torture allegations estimated in the model are conditional on country wealth and average levels of allegations in a country, via the random intercept. With judicial independence, it turned out that the multivariate effect estimates matched the unconditional, bivariate findings. This is not the case with democracy. 

Unconditionally, and going by allegation numbers, democracies only torture dissidents less than dictatorships. They are accused of torturing marginalized and criminal victims as much or even more than dictatorships are. Figure \ref{democracy} shows the average number of allegations for each of the six regime types in the Cheibub et all classification. Only for dissident victims are the averages when we compare the three democratic and dictatorship regime sub-types clearly lower. One necessary word of caution here is that while the averages differ, there is a lot of variation around these averages, i.e. differences between groups are outweighed by differences among individual countries and country-years. The absence of a bivariate relationship is especially surprising given that wealth decreases allegations, and democratic countries tend to be more wealthy.  

What do we make of this finding? As with judicial independence, they key might lie with changes over time versus differences between countries. Tentatively, it seems that countries that change to a democratic regime type experience a decrease in torture allegations against all victim types, but unconditionally democracies have as many allegations of torture of marginalized and criminal victims as dictatorships, and only lower levels of allegations that they torture dissidents. This makes sense in that dissent is a politically less threatening act there than in autocracies. What could reconcile the bivariate and multivariate findings is if democratic countries face unexpectedly higher levels of torture allegations given their wealth and that the random country intercepts pick up on this pattern. 

We did not specifically examine the impact of including the binary democracy indicator on model fit. Probably wealth accounts for most of the fit improvement if we compare models 1 and 2, and the XGBoost results also do not bode well. Logically, the ability of a binary indicator to predict count outcomes will be limited. 

\begin{figure}
\begin{center}
\caption{Bivariate relationships between latent judicial independence and torture allegations by victim type}
\label{fig:lji-bivariate}
\includegraphics[width=.7\textwidth]{../output/scatterplot-itt-allegations-v-lji.png}
\end{center}
\end{figure}

\begin{figure}
\begin{center}
\caption{Average number of torture allegations by victim type and regime type. Note that variation of cases around the averages is much larger than differences between the averages.}
\label{democracy}
\includegraphics[width=.7\textwidth]{../output/avg-allegations-by-regime.png}
\end{center}
\end{figure}

\textbf{Legal system}. The last factor we consider is legal system type. The \citet{Mitchell2013} classification has four categories: civil, common law, Islamic, and mixed systems. Civil law systems are the most common and we leave this as the reference category. Although the raw data hint at a higher level of allegations of torture of dissidents, and to a lesser extent criminal, in Islamic legal systems (Figure \ref{legal-system}), these differences are not very large and disappear in the multivariate estimates in model 4 in Figure \ref{fig:coefs}. Model fit also suggests that legal system types do not have much of a relationship with torture allegations. 

\begin{figure}
\begin{center}
\caption{Torture allegations by victim type accross legal system types}
\label{legal-system}
\includegraphics[width=.7\textwidth]{../output/boxplots-itt-allegations-v-legalsys.png}
\end{center}
\end{figure}

To summarize our model results, in our bivariate and coefficient estimates we find that judicial independence is related to a slightly lower number of allegations that dissidents are tortured. This is less true in the case of criminals, and not at all in the case of marginalized social groups. Although judicial independence is higher in democratic regimes, especially parliamentary democracies, its protective effect on the torture of dissidents is in addition to the reduction of dissident torture in democracies, and holds when accounting for regime type. In terms of model fit judicial independence predicts quite well between-country differences in all torture types, and outperforms democracy in predicting the torture of criminal suspects. Democracies face fewer allegations of torture of dissidents, but are similar to dictatorships in number of accusations that they torture marginalized groups and suspected criminals. There is some indirect evidence that changes to a democratic regime reduce torture allegations within a country. We find no evidence of a difference in torture allegations between civil, common law, Islamic, and mixed legal system.

Overall, our empirical results suggest that the institutional factors we examined--judicial independence, legal system types, and regime type more generally--explain a relatively small amount of the within-country variation in torture allegations against different types of victims. One caveat is that by including country random intercepts in our models, and thus accounting for between country variation, we have stacked the odds against finding strong associations. Institutional characteristics usually evolve relatively slowly over time, yet this is what our models concentrate on looking for. Indeed, if we were to exclude country intercepts, we would probably find stronger apparent relationships between these factors and torture allegation levels. The flip-side however, and benefit of the strategy we adopted, is that we can be more sure that what we have found, namely the protective effects of democracy and judicial independence on reduced levels of dissident torture, are not spurious results driven by unmodeled and unobserved differences between countries. With that in mind, one possibility we have not examined, but which may promise more clear results in respect to judicial independence and legal rules, is the hypothesis that these factors moderate the impact of adverse events on torture allegations. In other words, is it possible that legal and institutional protections against torture work best by moderating the ability of a state to adopt torture tactics when faced by domestic unrest or rebellion? 


Don't have cross-national data on all relevant features. But do have constitutional law. Several relevant ones. Barriers to arbitrary/lengthy state-imposed detention. Taking measures to prevent the state from detaining individuals for long periods of time should curb abuse. 

Haschke proxies social hierarchy with income inequality. We examine this by using EPR data and some stuff from V-Dem.   

VDem

EPR
epr is related to democracy. you cannot exclude a bunch of people and be a democracy. but exclusion in autocracies may not be related to ethnicity. only expect a lot of torture targeting marginalized groups if it is.  

CCP. does have info about whether coerced confessions are inadmissible, but gathered inconsistently.  

%%% Insert slick transition

Despite the limitations discussed above, we believe there are institutional safeguards that can mitigate oppressive violence. In fact, some of the institutions discussed above may be good for this purpose, but not for the reasons typically offered. That is, where oppressive violence is concerned, institutions matter, but not because they facilitate mass political action in response to abuse. Our argument is similar to that of \citet{ConradMoore2010}, who take a slightly different theoretical approach than that outlined in the previous section. Rather than imagining a leader who has to decide whether to abuse rights to achieve some goal, \citet{ConradMoore2010} imagine a leader who takes active measures to prevent abuse, or does not. They start from the premise that state agents will torture at least occasionally if they are not monitored and punished for doing so, and present an argument about how institutions affect a political leader's incentives to stop/prevent torture from occurring. They argue that democratic institutions matter not because public punishes leaders when torture occurs, but because democracy allows citizens to successfully pressure governments to create monitoring/oversight bodies that make torture less likely to occur in the first place. Our argument adopts their focus on the motives of state agents who engage in torture and the political leaders who try to encourage/discourage torture. Institutions can mitigate abuse because they create fewer opportunities and incentives for state agents to use torture, and increase the likelihood that those responsible for abuse are punished. Like Conrad and Moore, we start from the premise that individuals in the custody of the state are at automatically at risk for abuse.\footnote{Here ``custody'' means a state agent has limited an individual's liberty/freedom of movement for some period of time. A person in prison is therefore in the custody of the state, as is someone who has been told to sit on the sidewalk until further instructed.} State agents have various motives for abusing detainees, including interrogation, punishment/social control, and extortion. We do not expect institutions to trigger public outcries on behalf of abused detainees. Instead, we draw attention to legal institutions that help keep individuals out of custody for extended periods of time, and reduce incentives for agents to engage in abuse, either by creating oversight and punishment or by reducing the benefit of using torture on detainees. 

There are several existing studies that examine the relationship between legal institutions and personal integrity abuse, most of them focusing on judicial independence/effectiveness, constitutional provisions for individual rights, and legal system type  \citep{Davenport1996,Cross1999,Keith2002PRQ,Keith2002,KeithTatePoe2009,PowellStaton2009,RiosStaton2012,ConradRitter2013,Mitchell2013,Conrad2014,HillJones2014,RitterConrad2016}. One of our contributions is to develop an argument about, and examine, a broader set of legal rules that we think will be relevant for oppressive violence. We also consider oppressive and repressive violence separately, for the reasons discussed above, in order to examine the differential effects of these legal institutions on repressive and oppressive violence. Our argument is that legal institutions can limit oppressive violence because they 1) minimize opportunities for abuse by limiting the amount of time people spend in state-imposed detention, 2) create conditions conducive to monitoring the state agents most likely to engage in torture, 3) create justiciable rights that entail tangible penalties for individuals who commit abuse, and 4) reduce the incentive for police to use torture in the course of a criminal investigation. 

% work in cites of stuff on judicial independence, and constitutions. Keith, Powell/Staton, Keith/Tate/Poe, Davenport, Cross, Hill/Jones
Taking measures to prevent the state from detaining individuals for long periods of time should curb abuse. This is simply because there will be fewer opportunities for agents to engage in abuse. Many legal rules related to due process are designed to have this effect. Laws that prevent police from holding suspects in custody without a criminal charge, laws that allow for the possibility of pre-trial release, the right to a writ of {\em habeas corpus}, and and laws granting defendants the right to a fast trial create legal obstacles to lengthy detention. Due process rules related to arrest and trial procedure will also help reduce abuse. Here we have in mind laws that prevent police from making arbitrary arrests, as well as laws related to ``fair'' trial procedures. These would include general provisions for fair trials, the right to counsel, the presumption of innocence, and the right of defendants to examine witnesses who testify against them. The nature of the legal system may also be relevant. Legal scholars draw a distinction between adversarial and inquisitorial trial systems. In adversarial systems the court/judge takes no part in the criminal investigation, and instead serves as an impartial party to facilitate the presentation/examination of evidence by the prosecution and defense. In inquisitorial systems the judge does take part in the criminal investigation and may, for example, initiate the examination or presentation of evidence, including interviewing/questioning witnesses. 

Another way that institutions can discourage torture is by creating concrete penalties for those who engage in abuse. We argue that courts of law matter for this reason. Theories of coordination in response to government abuse seem to be concerned primarily with high courts that have the authority to make decisions about whether constitutional rules have been violated by government actors. Our concern, which is more general, is with the ability of courts of law to successfully prosecute/punish state agents responsible for abuse. Like previous studies, we argue that judicial independence/effectiveness can help reduce violations of personal integrity. This is both because independent/powerful courts will be more willing to prosecute state agents and, just as importantly, because strong courts create expectation that victims may actually seek redress \citep{PowellStaton2009}. In addition, we examine legal provisions that require prisons to keep registries which, where implemented, should make it more difficult for abuse to go undetected. 
%In addition to effective courts of law, we expect that laws explicitly prohibiting torture and related form of abuse will reduce the occurrence of violations. Of course, because formal rules exist does not mean they are followed or enforced. But formal rules create at least the possibility of punishment, and may heighten the expectation of legal punishment. Formal rules also give victims legal grounds to seek redress. Relevant rules here include explicit prohibitions on torture, cruel punishment, and corporal punishment, rules stating there can be no punishment without law, and laws granting victims the right to seek redress in the event that their rights are violated. 

Finally, we examine legal rules that reduce the benefit of torturing detainees to state agents. Extracting (false) confessions from criminal suspects is often a motive in police torture. As such, we expect that formal rules declaring coerced confessions to be inadmissible in criminal trials will decrease the police's incentive to use torture in the course of a criminal investigation. 

%Consider the recent spate of extralegal executions in the Philippines. Since coming to power in the Summer of 2016, President Rodrigo Duterte has implemented anti-drug policies that have resulted in thousands of illegal killings by the national police.\footnote{\url{https://www.hrw.org/news/2017/09/07/philippine-president-rodrigo-dutertes-war-drugs}} In fact, Duterte {\em campaigned} on his hard-line stance against drug users and sellers, and had developed a reputation for being ``tough on crime'' during his years as Mayor of Davao City. Though the killing of a high school student last summer caused some public backlash against these policies, the policies are still in place, and Duterte enjoys high public approval ratings according to the latest polls. 
%In his study of torture in democracies, \citet{Rejali2007} discusses police torture in the criminal justice system, arguing that it results from reliance on confessions rather than physical evidence in criminal trials, and legal rules that allow police to prolong pre-arrest detention. He also discusses violence that targets people who are not necessarily suspected of any criminal violation, but have transgressed customs concerning social hierarchies based on class, ethnicity, or some other social distinction. \citet{Rejali2007} refers to this as violence in pursuit of ``civic discipline.'' In the same vein as \citet{Walzer1973}, he describes an informal social arrangement between citizens and state agents where citizens demand ``social order'' and turn a blind eye to torture if they believe it accomplishes that goal.  

%Both incidents illustrate a fundamental fact about modern states; they possess tremendous coercive capacity and pose a potential threat to people who live in their territorial jurisdiction. Of course, states also serve the valuable purposes of creating basic social order and facilitating the provision of public goods. But any organization powerful enough to enforce laws and collect taxes also has the ability to use its coercive capacity against the public. In fact, the creation of social order and the extraction of wealth through taxation (even for public goods provision) are both inherently coercive, and abuses of authority may occur during the course of these activities. The question of how to design a government that wields tremendous coercive power but does not abuse that power is central to social contract theories of government, which date back at least to Hobbes. 

Despite this, most analyses use measures of state violence that do not distinguish between the two. When one reads through the sources that are typically used by political scientists to code data on state violence, it becomes apparent that many recorded instances of violence cannot be categorized as repression, as there is no clear political motive.\citep[][pp.\ 13--16, 90--91]{Haschke2018}. A systematic examination of the available data suggest that instances of oppression such as those recounted above are more common than incidents where people are targeted by state agents for their (perceived) political activities or affiliations Further, as we show below, these distinct types of state violence do not necessarily go hand in hand. This means that most existing explanations for state violence are intended to explain only a portion, perhaps a minority, of actual cases of violence, and are being tested using indicators that partly measure a phenomenon they are not intended to explain. 

One straightforward implication of the distinction between repression and oppression is that they may require different explanations. For example, political institutions associated with democracy, such as mass participation, competition, and constraints on executive discretion over policy (including an independent judiciary) are central in most studies of repressive violence. There are numerous studies that suggest state violence is less common in countries where these institutions are present \citep[e.g.,][]{Henderson1991,PoeTate1994,Davenport1996,Davenport1999,Poeetal1999,Keith2002,DavenportArmstrong2004,BdMetal2005,Davenport2007,ConradMoore2010}. However, recent research suggests that violent abuse of non-dissidents, i.e.\ oppression, is just as common in democracies as it is in non-democracies \citep{Haschke2018, JacksonHillHall2018}. These findings illustrate the need to distinguish between different forms of state violence. Most explanations for state violence, and the empirical models they motivate, may not provide much insight into patterns of oppressive violence. 

This study builds on recent work that examines repressive and oppressive violence separately. In the next section, we discuss further the distinction between repressive and oppressive state violence, and discuss some descriptive statistics and trends that speak to their prevalence and dissimilarity. 

For this purpose we leverage the Ill-Treatment and Torture (ITT) specific allegation data, which contains information about thousands of individual allegations of state torture. Whereas previous data on government violence did not distinguish between repressive and oppressive violence, the ITT data include information on the identity type of the victim, which allows us to examine the abuse of dissidents separately from other types of victims. After a brief examination of the data, we develop an explanation for why some political institutions may be more effective at discouraging repression than oppression. Domestic institutions are thought to limit abuse because they make political leaders accountable to the public and make it easier for the public to coordinate opposition to abuse, broadly defined. However, public backlash and coordination is less likely in response to cases of oppressive violence. We also broaden the scope of our inquiry to include legal institutions, which several studies have found to be important predictors of state violence \citep{Cross1999,Davenport2002,KeithTatePoe2009,Mitchelletal2013,HillJones2014}. Specifically, we examine constitutional prohibitions on torture, which have obvious relevance, and also constitutional provisions designed to limit arbitrary and lengthy state-imposed detention which places individuals at risk for abuse. Such legal guarantees, if they are enforced, reduce opportunities for abuse by state agents regardless of the potential victim's identity. Finally, we draw attention to the role that group-level social, economic, and political inequality play in state violence. We argue that oppressive violence is more likely where inequality based on class, ethnicity, or nationality is especially pronounced. We then conduct an analysis with the ITT data to examine whether political institutions, legal institutions, and group-level inequality have any meaningful relationship with oppressive or repressive violence. 



